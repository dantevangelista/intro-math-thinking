\documentclass[11pt]{exam}
\usepackage{amsfonts}
\usepackage{amsmath}
\usepackage{amssymb}
\usepackage{amsthm}
\usepackage{enumerate}
\usepackage{enumitem}
\usepackage{float}
\usepackage{geometry}
\usepackage{graphicx}
\usepackage{subcaption}
\usepackage{tikz}

\author{@dantevangelista}

\headrule{}

\begin{document}

\lhead{\textbf{Intro to Math Thinking Fall 2024: Assignment 4}}

\begin{enumerate}[leftmargin=0pt]

\item[1.]
\begin{displaymath}
    \begin{array}{c c c c c c}
        \phi & \psi & \phi \Rightarrow \psi & \psi \Rightarrow \phi & (\psi \Rightarrow \psi) \land (\psi \Rightarrow \phi) & \phi \Leftrightarrow \psi \\
        \hline
        T & T & T & T & T & T \\
        T & F & F & T & F & F \\
        F & T & T & F & F & F \\
        F & F & T & T & T & T
    \end{array}
\end{displaymath}

\item[2.]
\begin{displaymath}
    \begin{array}{c c c c c c c c}
          \phi & \psi & \neg \phi & \phi \Rightarrow \psi & \neg \phi \lor \psi & (\phi \Rightarrow \psi) \Rightarrow (\neg \phi \lor \psi) & (\phi \Rightarrow \psi) \Leftarrow (\neg \phi \lor \psi) & (\phi \Rightarrow \psi) \Leftrightarrow (\neg \phi \lor \psi) \\
          \hline
          T & T & F & T & T & T & T & T\\
          F & T & T & T & T & T & T & T\\
          T & F & F & F & F & T & T & T\\
          F & F & T & T & T & T& T & T
    \end{array}
\end{displaymath}

\item[3.]
\begin{displaymath}
    \begin{array}{c c c c c c}
        \phi & \psi & \neg \psi & \phi \nRightarrow \psi & \phi \land \neg \psi & (\phi \nRightarrow \psi) \Leftrightarrow (\phi \land \neg \psi) \\
        \hline
        T & T & F & F & F & T \\
        F & T & F & F & F & T \\
        T & F & T & T & T & T \\
        F & F & T & F & F & T 
    \end{array}
\end{displaymath}

\item[4.]
\begin{enumerate}[label=(\alph*)]
    \item \begin{displaymath}
    \begin{array}{c c c c c}
        \phi & \psi & \phi \Rightarrow \psi & \phi \land (\phi \Rightarrow \psi) & [\phi \land (\phi \Rightarrow \psi)] \Rightarrow \psi \\
        \hline
        T & T & T & T & T\\
        F & T & T & F & T\\
        T & F & F & F & T\\
        F & F & T & F & T
    \end{array}
\end{displaymath}
    \item it is a tautology meaning that $\psi$ always follows from knowing $\psi$ and $\phi \Rightarrow \psi$
\end{enumerate}

\item[5.]
$\phi \lor \psi$ means either $\phi$ or $\psi$ is true or both \\
Thus $\neg (\phi \lor \psi)$ means that $\phi$ and $\psi$ must both be false \\
This is the same as saying $\neg \phi$ and $\neg \psi$ must both be false (def of negation)\\
By def of $\emph{and}$, this can be written as $(\neg \phi) \land (\neg \psi)$

\item[6.]
\begin{enumerate}[label=(\alph*)]
    \item 34159 is not a prime number
    \item Roses are not red or violets are not blue
    \item There are no hamburgers but I won't have a hot-dog
    \item Fred won't go or he will play
    \item The number x is non-negative and less than or equal to 10
    \item We will lose the first game and the second
\end{enumerate}

\item[7.]
\begin{displaymath}
    \begin{array}{c c c c c c}
         \phi & \psi & \neg \phi & \neg \psi & (\neg \phi) \Leftrightarrow (\neg \psi) & \phi \Leftrightarrow \psi \\
         \hline 
         T & T & F & F & T & T \\
         F & T & T & F & F & F \\
         T & F & F & T & F & F \\
         F & F & T & T & T & T
    \end{array}
\end{displaymath}

\item[8.]
\begin{enumerate}[label=(\alph*)]
    \item \begin{displaymath}
    \begin{array}{c c c c c}
         \phi & \psi & \phi \Rightarrow \psi & \phi \Leftarrow \psi & \phi \Leftrightarrow \psi \\
         \hline 
         T & T & T & T & T \\
         F & T & T & F & F \\
         T & F & F & T & F \\
         F & F & T & T & T
    \end{array}
\end{displaymath}

    \item \begin{displaymath}
    \begin{array}{c c c c c}
         \phi & \psi & \theta & (\psi \lor \theta)& \phi \Rightarrow (\psi \lor \theta) \\
         \hline 
         T & T & T & T & T \\
         F & T & T & T & T \\
         F & F & T & T & T \\
         F & T & F & T & T \\
         T & F & F & F & F \\
         T & T & F & T & T \\
         T & F & T & T & T \\
         F & F & F & F & T
    \end{array}
\end{displaymath}
\end{enumerate}

\item[9.]
\begin{displaymath}
    \begin{array}{c c c c c c c c}
         \phi & \psi & \theta & (\psi \land \theta) & \phi \Rightarrow (\psi \land \theta) & \phi \Rightarrow \psi & \phi \Rightarrow \theta & (\phi \Rightarrow \psi) \land (\phi \Rightarrow \theta)\\
         \hline 
         T & T & T & T & T & T & T & T\\
         F & T & T & T & T & T & T & T\\
         F & F & T & F & T & T & T & T\\
         F & T & F & F & T & T & T & T\\
         T & F & F & F & F & F & F & F\\
         T & T & F & F & F & T & F & F\\
         T & F & T & F & F & F & T & F\\
         F & F & F & F & T & T & T & T
    \end{array}
\end{displaymath}

\item[10.]

$[\Rightarrow]$ Suppose $\phi$ is true \\
$\psi \land \theta$ means that both $\psi$ and $\theta$ must be true \\
Now suppose if $\phi$ is true, the $\psi \land \theta$ is true \\
By definition of \emph{implies}, this can be written as $\phi \Rightarrow (\psi \land \theta)$ \\
Since $\psi$ and $\theta$ are true, it follows: if $\phi$, then $\psi$; and if $\phi$, then $\theta$ \\
By def of \emph{implies} and \emph{and}, this can be written as $(\phi \Rightarrow \psi) \land (\phi \Rightarrow \theta)$ \\
\\
$[\Leftarrow]$ Suppose $(\phi \Rightarrow \psi) \land (\phi \Rightarrow \theta)$ is true \\
Then $\psi, \theta$ cannot both be false when $\phi$ is true \\
By def of \emph{and}, $\psi, \theta$ cannot both be false can be written as $(\neg \psi \lor \neg \theta)$ that is equivalent to $(\psi \land \theta)$ \\
Thus by def of \emph{implies}, the expression can be written as $\phi \Rightarrow (\psi \land \theta)$

\item[11.]
\begin{displaymath}
    \begin{array}{c c c c c c}
        \phi & \psi & \neg \phi & \neg \psi & \phi \Rightarrow \psi & (\neg \psi) \Rightarrow (\neg \phi) \\
        \hline
        T & T & F & F & T & T \\
        F & T & T & F & T & T \\
        T & F & F & T & F & F \\
        F & F & T & T & T & T
    \end{array}
\end{displaymath}

\item[12.]
\begin{enumerate}[label=(\alph*)]
    \item If 2 rectangles don't have the same area, they aren't congruent
    \item If $a^2 + b^2 \neq c^2$, then the triangle with sides $a, b, c$ ($c$ largest) is not right-angled
    \item If $n$ is not prime, then $2^n - 1$ is not prime
    \item If the Dollar does not fall, then the Yuan won't rise
\end{enumerate}

\item[13.]
\begin{displaymath}
    \begin{array}{c c c c c c}
        \phi & \psi & \neg \phi & \neg \psi & (\neg \psi) \Rightarrow (\neg \phi) & \psi \Rightarrow \phi \\
        \hline
        T & T & F & F & T & T \\
        F & T & T & F & T & F \\
        T & F & F & T & F & T \\
        F & F & T & T & T & T
    \end{array}
\end{displaymath}

\item[14.]
\begin{enumerate}[label=(\alph*)]
    \item If 2 rectangles have the same area, then they are congruent
    \item If $a^2 + b^2 = c^2$, then a triangle with sides $a, b, c$ ($c$ largest) is right-angled
    \item If $n$ is prime, then $2^n - 1$ is prime
    \item If the Dollar falls, the Yuan will rise
\end{enumerate}

\section{Optional Probs}

\item[1.] $\neg \psi \Rightarrow \phi$

\item[2.]
\begin{displaymath}
    \begin{array}{c c c}
         \phi & \psi & \phi \dot \lor \psi \\
         \hline
         T & T & F \\
         F & T & T \\
         T & F & T \\
         F & F & F
    \end{array}
\end{displaymath}

\item[3.] $\phi \dot \lor \psi$ is equivalent to $(\phi \land \neg \psi) \lor (\neg \phi \land \psi)$

\item[4.]
\begin{enumerate}[label=(\alph*)]
    \item If the statement is true, then it is not false
    \item If $x = 3$, then $x^2 = 9$
    \item If $-3 < x < 3$, then $x^2 < 9$
    \item If $x = 2$, then $x^2 = 4$
\end{enumerate}

\item[5.]
\begin{displaymath}
    \begin{array}{c c c c}
    M & N & M \times N & M + N \\
    \hline
    1 & 1 & 1 & 0 \\
    1 & 0 & 0 & 1 \\
    0 & 1 & 0 & 1 \\
    0 & 0 & 0 & 0
    \end{array}
\end{displaymath}

\item[6.]
\begin{enumerate}[label=(\alph*)]
    \item $\land$
    \item $\dot \lor$
    \item no
\end{enumerate}

\item[7.]
\begin{enumerate}[label=(\alph*)]
    \item $\lor$
    \item $(M \land N) \lor ( \neg M \land \neg N)$
    \item yes
\end{enumerate}

\item[8.] 2, cards B, 4

\item[9.]
Suppose $m,n$ are 2 natural numbers. If we multiply $mn$, then there are 3 cases:
\begin{enumerate}
    \item if at least 1 of $m, n$ is even, then $mn$ is even
    \item if both $m, n$ are even, then $mn$ is even
    \item if both $m, n$ are odd, then $mn$ is odd
\end{enumerate}
Then $mn$ is odd iff $m$ and $n$ are odd.

\item[10.] False, suppose $m$ is even and $n$ is odd, then $mn$ is even. Refer to statements in \emph{9.}

\item[11.] 1 face down ID, 1 7-up or vodka and tonic

\item[12.] Similar, need 2 verifications. Used contrapositive in Wason's problem and process of elimination in \emph{11.} Setup for \emph{11.} made it clearer.

\end{enumerate}
\end{document}